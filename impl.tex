\chapter{Implementation} % (fold)
\label{cha:impl}
	Just like in the previous chapter [\ref{cha:task}], every task will have its implementation process described in its own section.

	\section{Task 1} % (fold)
	\label{sec:impl_task_1}
		The implemented system consists of an UART Light peripheral as well as an I2C Master. More details concerning these two will be given in the following paragraphs. A third paragraph implements a filter which shall detect erroneous data.
	
		\subsection{UART Light} % (fold)
		\label{sub:uart_light}
			In JConfig, we selected the Uart Light Bus from the peripheral modules, setting a baudrate of 115200 and the same clock as the Spartan MC core. Using printf(), we sent the data measured by the sensor to the computer, where we read it from /dev/ttyUSB0.
		% subsection uart_light (end)

		\subsection{I2C Master} % (fold)
		\label{i2c_master}
			In this task, a basic implementation of I2C was designed, without making use of interrupts. First, I2C was added in JConfig and the SDA and SCL signals were connected to the pins of the FPGA according to the FPGA pin sheet.
			In order to fulfil the task, the following functions were implemented, besides the main function:
			
			\begin{itemize}
				\item sensor\_enable(i2c\_master\_regs\_t* i2c, int prescaler)
				\item sensor\_send(i2c\_master\_regs\_t* i2c, const unsigned char ch, const int slv\_addr, const int reg)
				\item sensor\_read(i2c\_master\_regs\_t* i2c, const int slv\_addr, const int reg)
		   	\end{itemize}
			
			\paragraph{Enable and Configuration} % (fold)
	   		\label{par:enable_task1}
   				The "sensor\_enable" function sets the bits of the control register, in order to establish a correct I2C connection. The I2C core is enabled by setting the CORE\_EN bit. Additionally, the right frequency of the SCL line is chosen by assigning the prescaler with the correct value. In the main function, this corresponds to a frequency of \SI{40}{\kilo\hertz}.
			% paragraph enable_task1 (end)
				
			\paragraph{Data Write} % (fold)
	   		\label{par:data_write_task1}
	   			A write access, which occurs when calling the "sensor\_send" function, must contain the id of the slave, the register number where the data will be written and the data to be sent. The transfer is initiated by setting the STA bit and the number of transferred bytes. Counting the number of bytes sent as useful data is not enough. The COUNT value takes into consideration also the bytes containing the slave address and the register number.
				We use this function to tell the sensor to start ranging in centimetres. This means that we write the value 81 in the register 0.
			% paragraph data_write_task1 (end)
			
			\paragraph{Data Read} % (fold)
	   		\label{par:data_read_task1}
	   			The read access starts with a write of two bytes, telling the sensor the register number from which we want to read. Afterwards, a read request is initiated, by setting the last bit of the slave id. Additionally, the RD bit of the Command register is 1. The COUNT value is now 3, which means that 1 byte will be written (the slave id) and 2 bytes are requested. These 2 come from the fact that the result is split in the registers 2 and 3 of the sensor and has 16 bits. The result is than read from the data bits of the I2C register and the STO bit of the command register is set.
			% paragraph data_read_task1 (end)
			
			\paragraph{Main function} % (fold)
	   		\label{par:data_read_task1}
	   			As stated above, the main function sets a frequency of 40kHz for the SCL line, by calling the sensor\_enable function. It sends to the sensor the command to start measuring in centimetres. A delay was implemented to wait for the SRF02 to finish ranging and afterwards the result is read through the I2C and sent to the computer via the UART Light. This is done in an endless loop.
			% paragraph data_read_task1 (end)
		% subsection i2c_master (end)

		\subsection{Filtering of acquired data} % (fold)
		\label{sub:filter}
			In order to filter the information obtained from the distance measuring sensor, a comparison to the before received data must be conducted. Thus, storing part of the measured data is necessary.
			This project proposes a simple implementation of such a filter by storing ten measured values and comparing their mean to the new one. If the latest measurement is greater or smaller than the mean by 50 units, than the value is not printed. However, it is stored. When the program starts, no value is already available, thus the first one is printed without filter. In case the first value was incorrect, the filter will converge anyway to a good result after some measurements.
			This approach was easy and fast to implement and returns good results for this type of application. If a high reliable result was required, then the Kalman filter could have been a good replacement. In this case, the behaviour of the sensor must be described.
		% section filter (end)
	% section task_1 (end)

	\section{Task 2} % (fold)
	\label{sec:impl_task_2}
		Communication with the OLED display is to be facilitated by using an SPI master, which has to be added to the jConfig project. Connecting the SPI master instance to the required FPGA pins is straight forward as the required connections (CS, SCLK, SDIN, RES) are denoted by the the provided FPGA pin sheet. However, the displays' reset pin requires a separate output port instance to be generated by jConfig and can not be connected directly to the SPI paster peripheral. 

		\subsection{SPI Implementation} % (fold)
		\label{sub:impl_spi_implementation}
			Implementing the SPI driver is split into three parts:

			\begin{itemize}
				\item SPI enable and initial configuration
				\item SPI slave de-/select
				\item SPI data write
	   		\end{itemize}

	   		Each listed item will be described in one of the following paragraphs.

	   		\paragraph{Enable and Configuration} % (fold)
	   		\label{par:enable}
	   			Enabling and configuring the SPI master is implemented by writing to its control register. The task description states that the serial clock of the SPI connection has to be set to under \SI{10}{\mega\hertz}. This is achieved by setting the clock divider to two, resulting in a frequency of \SI{7.5}{\mega\hertz}. The bit width for each transfer is retrievable from the displays SPI manual and the fact that we will be using the described three wire interface, relying only on the serial data pin to send commands and data to the display. Reading further in the manual shows that eight bits plus one mode selection (command/data) bit are part of each transfer, making this nine bit in total. As a result of these observations, the value nine is written to the BITCNT property of the control register. Lastly, the enable bit is set, completing the initial configuration procedure.
	   		% paragraph enable (end)

	   		\paragraph{Slave De-/select} % (fold)
	   		\label{par:slave_de_select}
	   			Selecting the SPI slave targeted by the master requires a read to its control register. As the only available slave is attached to the first slave select signal, this value will be set one time before using the SPI interface and remains the same for the duration of the whole program.
	   		% paragraph slave_de_select (end)

	   		\paragraph{Data Write} % (fold)
	   		\label{par:data_write}
	   			As there is no need to retrieve data from to display, the write implementation will suffice to be able to use it. Writing to the masters data output register and waiting for all bits to be send is the only thing to do. This first (but also the final interrupt driven) implementation is polling the fill bits of the status register until the retrieved value is 0, denoting an empty output register and the ability to send more data.
	   			It is notable that the nine bits (one selecting command/data mode and eight actual with payload) to send are composed in such a way that the mode selection bit is the highest one.
	   		% paragraph data_write (end)
		% subsection impl_spi_implementation (end)

		\subsection{Display Driver Cleanup} % (fold)
		\label{sub:impl_display_driver_cleanup}
			As source code for the display driver is already provided, it remains to adjust it to our system. This includes the replacement of the data and command transaction routine implementations with own code calling the SPI procedures, but extends to the not compiling driver code. The latter is caused by multiple missing type declarations of local variables but also by the not defined global variable "Shift". Further errors are generated by the missing inclusion of the font header and by undefined GPIO related macros. As all GPIO accesses can be replaced by SPI or the newly introduced reset output port, no undefined variables and macros remain.

			\paragraph{ASCII Table} % (fold)
			\label{par:ascii_table}
				The recently mentioned font header provides data for printing ASCII characters to the display. However, when trying to write string literals to the display it can be observed that the provided table is shifted by an offset of minus 32. Simply adding this value to each index used in the Show\_Font57\_25664 procedure resolves this issue.
			% paragraph ascii_table (end)

			\paragraph{Global Shift} % (fold)
			\label{par:global_shift}
				A global variable called "Shift" is used to place correctly the text on the horizontal axis of the display. Experimentally, we set this variable to the value 28. 
			% paragraph global_shift (end)
		% subsection display_driver_cleanup (end)

		\subsection{Display Usage} % (fold)
		\label{sub:display_usage}
			Before using the display it is required to configure the SPI master and initialise the display with the already available OLED\_Init\_25664 procedure. Afterwards, the displays' RAM is initialised to ensure a none noisy background.
			Printing strings to the display is achieved by a call to Show\_String\_25664. At this point it is possible to combine the sensor data retrieval with the display and solve the third task.
		% subsection display_usage (end)
	% section task_2 (end)

	\section{Task 3} % (fold)
	\label{sec:impl_task_3}
		We are now able to print strings on the OLED display and to read data measured by the sensor. The updates required to combine these functionalities are presented in the following chapters.
		
		\subsection{Integer to String Conversion} % (fold)
		\label{sub:integer_to_string_conversion}
			The data read from the sensor was stored as an integer value. To print it on the OLED display, we implemented a function that converts an integer to a string: "int\_to\_str". Each cipher of the number is evaluated, and a string is constructed accordingly. The function can convert a number less than 1000. If the first cipher is a zero, it is replaced by a space character when building the string. Since the sensor can measure at least 15cm, there is no need to replace the zeroes on the second position.  
		% subsection integer_to_string_conversion (end)

		\subsection{Integration} % (fold)
		\label{sub:integration}
			The two functionalities are combined in a new main function. The SPI peripheral, namely the OLED display, is enabled and a delay is inserted to wait for this process to complete. Some data regarding the control and status registers is sent via UART for debug purposes. After the SPI slave is selected, the display and it’s RAM are initialized by calling the "OLED\_Init\_25664" and "Clear\_RAM" procedures. Using the function "Show\_String\_25664" we print the text "Der Abstand ist: " and emphasize it through some dashed lines. 
			In an endless loop, the function "sensor\_read" is called. This starts a new measurement in centimeters, and after a delay of approximately 1000*6000*3 clock cycles, which results in a delay of around \SI{300}{\milli\second}, the result is read from the sensor. This delay is sufficient for the sensor, which according to the datasheet is able to measure in \SI{65}{\milli\second}. To save some time, we can reduce the value given to the sleep function by 3 (1000*2000*3 clock cycles). The value read from the sensor is converted to a string and is printed on the display. It comes after the text "Der Abstand ist: ", thus an adjustment of the horizontal offset was needed.

		% subsection integration (end)
	% section task_3 (end)

	\section{Bonus Task 1} % (fold)
	\label{sec:impl_bonus_task_1}
		Starting with initial thoughts on possible gains and costs for choosing SPI or I2C as the optimisation target, this section describes how the first bonus task was solved and which problems were encountered.

		\subsection{Initial Considerations} % (fold)
		\label{sub:initial_considerations}
			The first important question for solving this task is which components may profit from an interrupt driven operation. Both I2C and the SPI master interfaces are associated with waiting times when using them. 

			Updates of the display are limited to three bytes in size as measurements in cm will not produce results with more than three digits when considering the capabilities of the ultrasonic sensor. Furthermore, communication with the display (using the SPI interface) only has to occur when there is a change in the data to display. Busy waiting a small amount of cycles for these three bytes to be send seems to be preferable to the additional complexity introduced by implementing an interrupt based solution. Additionally, as section [\ref{sub:impl_spi_implementation}] shows, the SPI interface is working at clock frequency ranges up to \SI{10}{\mega\hertz}. This is still slower than the processor clock (\SI{60}{\mega\hertz}) but the difference does not span multiple orders of magnitude. 
			All this shows that it is not worth while to change the display update procedures to be interrupt driven.

			Read and write transaction sizes for accessing the ultrasonic sensor are comparable to the display update sizes. On the other hand, considerable waiting time is associated with retrieving the measurement results. As the I2C interface operates at a frequency of \SI{40}{\kilo\hertz}, it is possible to implement a feasible interrupt based sensor write and read process which actually provides large performance benefits. With this in mind, support for interrupt based I2C operation was implemented.
		% subsection initial_considerations (end)

		\subsection{Additional Hardware} % (fold)
		\label{sub:additional_hardware}
			Reading the SpartanMC manual provides basic information about the simple interrupt controller which is mandatory to be able to use interrupts in the SpartanMC context. Activating the interrupts on the I2C master and connecting its interrupt output port to the interrupt controller completes the peripheral side of the required hardware changes. Afterwards the interrupt signal of the controller is connected to the SpartanMC processor. At this point it remains to implement the interrupt service routine (ISR).
		% subsection additional_hardware (end)

		\subsection{Measurement FSM and Interrupt Handling} % (fold)
		\label{sub:measurement_fsm_and_interrupt_handling}
			To accommodate the different states of setting up a new measurement and retrieving the result, a simple state machine is required. We choose to implement state changes and state performed operations in the main loop to reduce the size and complexity of the ISR.
			A signal called change\_state is set each time an interrupt occurs. In an endless loop, the main function evaluates this signal. When equal to one, the current state is evaluated and the change\_state signal is reset, so that no further changes of the state will occur without an interrupt.

			\paragraph{States and Transitions} % (fold)
			\label{par:states_and_transitions}
				% state and state changes
				The reset state is INIT\_STATE and it starts a new measurement in centimetres. An interrupt occurs when this control data was sent and afterwards the next state, "WAITING\_FOR\_MEASURMENT\_REQUEST\_SEND", is evaluated. In this step, a transfer requesting the firmware version is conducted. In the next phase, the acknowledge bit is checked. If it is 0, an I2C read access occurs. Otherwise, the read request is sent again. When reading the firmware version, 0xff means that the sensor is currently ranging and ignored the I2C request. The state machine does a polling until the result is available. When this happens, a read request starting from register two is sent. If no error occurred in the transfer, a read access of two bytes is conducted. In the "WAITING\_FOR\_REQUEST\_RECEIVE\_DATA" state, the value is read from the data register, it is filtered and if valid, it is printed on the OLED display.
			% paragraph states_and_transitions (end)
			
			\paragraph{Causes for Interrupts} % (fold)
			\label{par:causes_for_interrupts}
				When implementing the solution containing interrupts, a difficulty was understanding when an interrupt occurs. To avoid any possible errors, we looked directly in the Verilog files. From the code we understood that the interrupt flag is reset when the I2C module is addressed. Thus, we decided to read the status register when entering the interrupt routine. Only after a full byte was sent could an interrupt occur and even then, just in certain cases. One example was when the NO\_ACK bit of the status register was one. In the beginning we expected an interrupt only after a full transaction was completed, so we omitted this case. After including it in the state machine, we obtained a functional code.  
			% paragraph causes_for_interrupts (end)
			
			\paragraph{The Interrupt Service Routine} % (fold)
			\label{par:impl_isr}
				The interrupt procedure is called automatically when an interrupt occurs. It has the main purpose of setting the "change\_state" signal, described above. It also reads the status register, because addressing the I2C module resets the interrupt flag.
			% paragraph impl_isr (end)

			\paragraph{Required Workarounds} % (fold)
			\label{par:required_workarounds}
				The SRF02 sensor needs around \SI{65}{\milli\second} in order to complete a measurement. In the original solution, we implemented a simple delay in order to overcome this issue. However, after designing a finite state machine to work with interrupts, we removed these delays. Thus, we sometimes got unrealistic values from the sensor. This was because the sensor did not finish ranging.
				In order to overcome this issue, we extended the initial state machine in a way that before requesting a result, the software version from register 0 is required. According to the data sheet of the sensor, is the measurement did not complete, the device will not respond to I2C request and the I2C Master will receive 0xff on the data bus. A value different form 0xff signals that the ranging is complete, so we continue with requesting the result from the registers two and three.
			% paragraph required_workarounds (end)

			% - state changes initiated by boolean global
			% - boolean value switched when ISR is called
			% - check the new state of the code
		% subsection measurement_fsm_and_interrupt_handling (end)
	% section bonus_task_1 (end)

	\section{Bonus Task 2} % (fold)
	\label{sec:impl_bonus_task_2}
		% - use autoperf
		The third task was fulfilled using the performance counter. This was added in the jcnofig tool, by selecting this feature from the parameters of the "spartanmc\_0". In order to evaluate the sleep() function, one cycle counter was used for which the prescaler variable was set to zero. The following sequence of functions was implemented: perf\_init(\*), perf\_start(\*), sleep(\*), perf\_stop(\*), perf\_read(\*), perf\_results\_printf(\*), perf\_reset(). The following values were obtained when profiling the sleep function:
		\begin{itemize}
			\item Sleep(5000)  lead to 15012 cycles
			\item Sleep(5000)  lead to 15012 cycles
		\end{itemize}
		According to the description of this function, it returns a delay of 6 \+  3 \* input\_value clock cycles. In the first example, 6 cycles more are needed while in the second one, extra 5 cycles occur. This could be due to ... \textbf{I will take another look at the asm of sleep}
		% - asm of sleep and sleep_sleep provides a good way to prove the execution characteristics 
	% section bonus_task_2 (end)
% chapter impl (end)
