\chapter{Introduction}
\label{cha:intro}
	During the lecture "Rechnersysteme II" students are able to take part in an exercise about System on Chip (SoC) Kits. These kits provide the user with a tool chain to seamlessly integrate software and hardware components in a single system on a single chip. The SoC Kit in use is based on the SpartanMC processor, which was developed to be implemented in the context of FPGAs making optimal use of their resources.

	The main assignment of the exercise is to implement a distance meter using an ultrasonic sensor and an OLED display to visualise the measured data. Multiple tasks are provided to get to know and integrate the required peripherals step by step. Software has to be written for communicating with the ultrasonic sensor (SRF02) and the OLED display, which are integrated into the overall system by using jConfig. Each peripheral uses its own serial bus interface (I2C and SPI).

	Bonus tasks revolve around implementing an interrupt based solution for the bus interface drivers of the ultrasonic sensor and the OLED display. Further more, code profiling of two procedures and verification of their runtime behaviour is part of the set of additional tasks.

	All tasks, including the bonus tasks, of the exercise were handled during the curse of solving it.