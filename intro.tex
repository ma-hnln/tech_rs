\chapter{Introduction}
\label{cha:intro}
	The second exercise of the lecture Rechnersysteme II revolves around the implementation of a software solution for acquiring satellite signals in the Global Positioning System (GPS). Target hardware is the AMIDAR processor, with its dynamic hardware acceleration capabilities based on the CGRA concept. The later sections of this chapter provide a brief overview of both AMIDAR and the CGRA.

	The main assignment of the exercise is to implement a working GPS acquisition algorithm and then optimise it in two dimensions: performance and energy consumption. Both parameters are going to be measured by the AMIDAR simulator. Detailed descriptions for both sub tasks can be found in the following chapter [\ref{cha:task}], while the implementation details are deferred to chapter [\ref{impl}].

	\section{GPS Acquisition} % (fold)
	\label{sec:intro_acq}
		The first step in being able to make use of GPS is to acquire satellite signals such that the own position may be tracked by using the information provided by multiple satellites.
		Focus of this exercise are the L1 carrier bands' (\SI{1575}{\mega\hertz}) Coarse-acquisition (C/A) codes. GPS is using a Code Division Multiple Access scheme to discriminate the different satellite signals. The C/A codes are therefore different for each satellite and are locally available to the receiver. 
		Discrimination of the different satellites is performed under the assumption that the C/A codes are orthogonal codes. This means that the cross correlation of unequal C/A codes yields a minimal result while the auto correlation is yielding a maximum result. Correlation is expressed in terms of the convolution of the two signals, which in turn means that it is possible to use the Discrete Fourier Transform to solve this problem by using element wise multiplication in the frequency domain. This approach was presented to us with the exercise material and is going to be used by our implementation.
	% section intro_acq (end)	

	\section{AMIDAR} % (fold)
	\label{sec:intro_amidar}
		The Advanced Micro Instruction Driven Architecture (AMIDAR) models a dynamically reconfigurable processor which uses multiple functional units (FU) to execute given instructions. 
		To do so, instructions are split in multiple operations which are then distributed to their respective functional units. As any computation results are tagged and are therefore identifiable by the FUs, synchronisation of the executed operations can be maintained in a very simple manner. 
		In principle, any type of FU is possible, as long as it adheres to the requirements of the token network, serving as the interconnect of all FUs. This includes an implementation of a CGRA which is configured at runtime by profiling the executed code and scheduling recurring instruction sequences to be executed on it.
	% section intro_amidar (end)

	\section{CGRA} % (fold)
	\label{sec:intro_cgra}
		The Coarse Grain Reconfigurable Array (CGRA) is the targeted hardware accelerator of this task. Like FPGAs, CGRAs can be reconfigured at runtime but are targeted at word level usage, like arithmetic operations on \num{32} bit values. This reduces the required amount of configuration data (the so called Context) which in turn makes it possible to store more of it, resulting in the ability to change the configuration of individual elements with every clock cycle.
		Multiple Processing Elements (PE) form the back bone of the CGRA. Each comes with its own Context memory which configures its remaining components. PEs hold a register file (RF), an ALU and multiple multiplexers which control the data flow into, out of and inside the PE. A usable array of PEs is required to feature a connection scheme which enables neighboring PEs to work together.
		Execution of the whole array, meaning which contexts are used in a given clock cycle, is controlled by the Context Control Unit (CCU). The CCU generates a context counter (CCNT, like a program counter) which is routed to all CGRA components. For conditional branching, the Condition Box (C-Box) is used to determine if the branch is to be taken or not. This is done by evaluating the states of the PE ALUs.
		Generally and in the context of our task, there are no restrictions imposed related to the structure of the CGRA. This means that inhomogen, where PEs are able to execute different sets of instructions, and irregular, where PEs are connected in an arbitrary way, CGRA compositions are allowed and usable.
	% section intro_cgra (end)

% chapter intro (end)