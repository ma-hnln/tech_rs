\chapter{Task Description}
\label{cha:task}
	This chapter provides an overview of the two tasks which are part of the exercise.
	First and foremost, it is required that a working implementation of the GPS algorithm is implemented which is going to be the basis for all other optimisations. This initial solution is required to run on the AMIDAR simulator, which is directly executing the Java code we write.

	\section{Algorithm} % (fold)
	\label{sec:task_algorithm}
		For implementing the initial solution, we are provided a version of the algorithm for GPS acquisition. The general working of it can be described as follows.
		Taking in the locally generated (complex) samples of one C/A code and a set of received (complex) samples of the same size like the given C/A code, the algorithm determines if the given samples match the given C/A code. 
		To do so, some more configuration must be provided. As satellites are constantly in motion, changing the distance to the receiver, Doppler-Shift has to be taken into account. As it is not known if the satellite is moving closer or further away when acquiring the signal, a range of positive and negative Doppler-Frequencies has to be checked. With the given sample input and configuration in terms of the Doppler-Frequencies, acquisition starts by wiping off the carrier from the received signal. This initial step is preformed for every configured Doppler-Frequency and requires that all further steps are also performed for each of its results. 
		For cross correlating the code samples with the received samples, we apply the DFT to both and multiply them. This multiplication is performed element wise on the received sample DFT result and the complex conjugated DFT result of the code samples.
		After applying the inverse DFT, to the sample vector resulting from the latest multiplication, the vector is searched for its maximum value. As complex values do not confer meaning to minimum and maximum, only the absolute value of the complex number is considered when searching the maximum. The position of the found maximum is treated as the phase shift of the code in question. 
		Remembering that we have to execute the previous steps for each Doppler-Frequency, another component of the result is the one Doppler-Frequency which yielded the maximum absolute correlation result. 
		To be able to determine if the found maximum is high enough to be sure that the received signal matches the given code, the received signals power has to be calculated and set into relation with the found maximum value. If this relation exceeds the acquisition threshold, the received signal matches the given code samples and the corresponding satellite is now acquired.
	% section task_algorithm (end)

	\section{Maximum Performance} % (fold)
	\label{sec:task_max_perf}
		The first and straight forward goal of this exercise is the optimisation of the Acquisition algorithm in the performance domain. Mathematical and code optimisations are allowed, just like using a completely different CGRA than the initial one. Parameters of the scheduler are also changeable.
		The used metric for measuring the results is the cycle counter of the AMIDAR simulator.
	% section task_max_perf (end)

	\section{Minimum Energy Consumption} % (fold)
	\label{sec:task_min_energy}
		Featuring the same rules than the first task, the focus of this task shifts to an energy optimal solution. As the AMIDAR simulator is also tracing energy usage during execution, this will be the used metric.
	% section task_min_energy (end)
% chapter task (end)
